\documentclass{article}
\usepackage[margin = 0.15in,landscape]{geometry}
%\usepackage{wasysym}
\usepackage{multicol}
\usepackage{array}
\usepackage{amsmath}
\usepackage{amssymb}
\usepackage{lmodern}
\usepackage{graphicx}
\usepackage{enumitem}
\usepackage{stix}
\usepackage{mathrsfs}
\setlength\parindent{0pt}
\renewcommand{\baselinestretch}{0.75}

\begin{document}
\begin{multicols*}{3}
    Marissa Palamara\par 
    ASTR 3750\par 
    Fall 2021\par 
    Quizes 1 \& 2
    \vspace{-0.5cm}
    \setlist{nolistsep}

    % ----- Overview ----- %
    \section*{Planets, Moons, and Rings Overview}
    \subsection*{Planets}
    \textbf{Terrestrial}\par 
    \begin{itemize}
        \item Smaller bodies made mostly of rock and metal
        \item Composition dominated by Si and O
        \item Atmosphere $< 1/1000^{\text{th}}$ mass of planet 
        \item Natural satellites and dwarf planets are all terrestrial 
        \item Planets
        \begin{itemize}
            \item Mercury
            \item Venus
            \item Earth
            \item Mars 
        \end{itemize}
        \item Dwarf Planets
        \begin{itemize}
            \item Ceres
            \item Pluto
            \item Haumea
            \item Makemake
            \item Eris
        \end{itemize}
        \item Exoplanets
        \begin{itemize}
            \item Super-Earths
            \item Mini-Earths
        \end{itemize}
    \end{itemize}
    \textbf{Jovian}
    \begin{itemize}
        \item Giant Planets
        \item Composition dominated by H, He, C, N, O
        \item Massive balls of gas with ice/rock/metal cores
        \item Atmospheres comparable to mass of planet 
        \item Gas Giants
        \begin{itemize}
            \item Jupiter
            \item Saturn
            \item Hot Jupiters (Exo)
            \item Cold Giants (Exo)
        \end{itemize}
        \item Ice Giants
        \begin{itemize}
            \item Uranus
            \item Neptune
            \item Mini-Neptunes (Exo)
        \end{itemize}
    \end{itemize}

    % ----- Star and Planet Formation ----- %
    \section*{Star and Planet Formation}
    \textbf{Raw Materials Abundance:} Most to Less
    \begin{itemize}
        \item H
        \item He 
        \item O
        \item C
        \item N, Ne
        \item Mg, Si, Fe 
        \item S, Ar
    \end{itemize}
    \textbf{Protostellar Nebula}
    \begin{itemize}
        \item Gas Density
        \begin{itemize}
            \item $100-1000 \text{cm}^-3$
            \item Cores $10^3-10^6 \text{cm}^-3$
        \end{itemize}
        \item Starting to collapse under own gravity
    \end{itemize}
    \textbf{Contraction}
    \begin{itemize}
        \item ~100,000 years
        \item Gas drag damps oscillations about mid-plane 
    \end{itemize}
    \textbf{Condensation}
    \begin{itemize}
        \item Critical moment for planetary composition
        \item Coagulation to ~1m sized particle in ~1 Myr 
    \end{itemize}
    \textbf{Accretion}
    \begin{itemize}
        \item Orderly: ~100km in ~1Myr
        \item Runaway: gravitational focusing forms embryos/protoplanets ~0.1-10 M$_\text{Earth}$
        \item Oligarchic: a few rules dominate their feeding zones
        \item Critical size to accrete non-condensables (H,He): ~10 M$_\text{Earth}$
    \end{itemize}
    \textbf{Clearing}
    \begin{itemize}
        \item Jovian planets tug on smaller bodies
        \item Migration
        \item Late Heavy Bombardment
    \end{itemize}

    \subsection*{Frost-Line}
    \begin{itemize}
        \item Within the frost line, rocks and metals condense while H compounds stay gaseous
        \item Beyond, H compounds, rock, and metals condense 
        \item H and He themselves don't condense anywhere
    \end{itemize}    
    


\end{multicols*}
\end{document}
